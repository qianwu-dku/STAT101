\PassOptionsToPackage{unicode}{hyperref}
\documentclass[11pt]{article}
\usepackage[margin=1in]{geometry}
\usepackage[pdftex]{graphicx}
\usepackage{multirow}
\usepackage{setspace}
%\usepackage{url}
\pagestyle{plain}
\setlength{\parskip}{0.6em} 
\setlength\parindent{0pt}

\usepackage{footnote}
%\usepackage{tablefootnote}
\makesavenoteenv{tabular}
\usepackage[dvipsnames]{xcolor}
\definecolor{darkblue}{RGB}{0,0,128}

\usepackage[hyphens]{url}
\usepackage{hyperref}
\hypersetup{
    colorlinks=true,
    linkcolor=darkblue,
    filecolor=magenta,      
    urlcolor=darkblue,
    pdftitle={Overleaf Example},
    pdfpagemode=FullScreen,
    }
\usepackage{enumitem}
\setlist{itemsep=-0.1em}
\usepackage{scrextend}

\usepackage{array}
\newcolumntype{L}[1]{>{\raggedright\let\newline\\\arraybackslash\hspace{0pt}}m{#1}}
\newcolumntype{C}[1]{>{\centering\let\newline\\\arraybackslash\hspace{0pt}}m{#1}}
\newcolumntype{R}[1]{>{\raggedleft\let\newline\\\arraybackslash\hspace{0pt}}m{#1}}

\begin{document}

% Course information
\noindent \hspace{-4mm} \begin{tabular}{ l c l }
  \multirow{3}{*}{\includegraphics[scale=.5]{logo.jpeg}} && \Large\textbf{STATS 101 - Introduction to Applied Statistical Methods}\\[5pt]
  && \Large \textbf{Fall 2025, Session 2}\\[24pt]
  && click \href{https://qianwu-dku.github.io/STAT101/syllabus/syllabus.pdf}{here} for latest version \\
  && Section 1434 (lecture)\\
  && Section 1435 (lab)\\\\
\end{tabular}
%\vspace{2mm}

% Professor information
\noindent  \hspace{-4mm}  \begin{tabular}{ l l }
  \textbf{Instructor:} 			& \large Qian Wu \\
  \textbf{Office Location:} 	& \large IB2017\\
  \textbf{E-mail:} 				& \large \href{mailto:qian.wu2@dukekunshan.edu.cn}{qian.wu2@dukekunshan.edu.cn} \\\\
 
  \textbf{Class Dates:}			& 10/20/2025 – 12/04/2025 \\
  \textbf{Lecture Time:}			& \large MoWe 12:00PM - 2:30PM\\
  \textbf{Lecture Location:} 		& \large AB 1087\\
    \textbf{Office Hours:} & \large MoWe 9:30AM -- 11:30AM and by appointment\\
     & \\
 \textbf{Lab Manager: }          & \large{Jingyu Wang} \\
\textbf{E-mail: }          & \large{\href{mailto:jingyu.wang@dukekunshan.edu.cn}{jingyu.wang@dukekunshan.edu.cn}} \\
 \textbf{Lab Time:}              & \large Tu 1:30PM - 2:30PM\\
 \textbf{Lab Location:}          & \large IB 2028\\
 \textbf{Office Hours:} & \large Fr 10:00AM - 12:00PM in WDR 2134\\



\end{tabular}




\vspace{1.5cm}
\large \textbf{Course Description:}\\[0.5em]
How can we use data to shed light on age-old and new human problems such as pollution, discrimination, and economic growth? How can we be “sure” that the evidence we have points us in the right direction? How meaningful are our findings? Do our results suggest the relationships we find between factors such smoking and cancer are meaningful or meaningless? How would we know? How should one properly display and explain your statistical results to these important issues?

This class introduces you to the tools and concepts that begin to tackle these questions. We will cover topics such as data summaries, sampling, data analysis, production of graphical displays, and regression techniques. The goal at the end of the course is that you will be able to conduct basic data manipulation, know how to properly summarize and display data, and make basic statistical inferences using real datasets.

A third set of goals for the course is that you will also be able to read more fluently research literature that employs statistics. During this course I will reference, in class, a number of historically important academic articles and we will analyze the data from those articles. Doing so should help you understand how data is used (and misused) to construct social science arguments.


\vspace{0.5cm}
\large \textbf{Required Background:}\\[0.5em]
The emphasis in the course will not be on learning mathematical formulas related to statistics but rather to develop an intuitive understanding of statistical inference and measures of uncertainty. For those interested in developing this intuitive understanding of statistics in a more rigorous way, you may also consider taking the following courses:
\begin{itemize}
    \item MATH 205: the mathematical foundations of statistics.
    \item ECON 203: advanced study of modern regression techniques.
    \item SOSC 320: advanced statistical techniques applied to real-world problems.
\end{itemize}






\vspace{0.5cm}
\large \textbf{Course Objectives:}\\[0.5em]
Upon completing the course, you will develop the following abilities:
\begin{itemize}
    \item Intuitively interpret statistics in course materials and in the larger world.
    \item Become a statistics results producer in addition to a statistics consumer.
    \item Assess when and how to use statistics to answer specific questions in the social sciences.
    \item Analyze how previously learned problems can be answered with statistical methods.
    \item Apply statistical methods to future social science coursework and capstone project.
    \item Judge how appropriately statistics are used in everyday life when reading the news, business reports, and other real-world applications.
\end{itemize}

In support of this you will be able to:
\begin{itemize}
    \item Understand and interpret basic statistical properties of data (confidence intervals, t-tests, etc).
    \item Identify when various statistical tests are appropriate given a specific dataset.
    \item Formulate testable hypotheses in the data and learn how to execute those tests.
    \item Interpret statistical results to understand both significance of the results and their substantive impact.
    \item Illustrate statistical results with appropriate and clear graphical displays that provide meaning to the reader.
    \item Evaluate critically other, published, statistical work with the skills and techniques learned in class.
    \item Propose an independent research project that integrates statistical methods with their research interest for their capstone project.
\end{itemize}





\vspace{0.5cm}
\large \textbf{Course Structure:}\\[0.5em]
In general, each week will proceed roughly as follows:
\begin{itemize}
    \item Monday-Thursday: Read the textbook chapter, make progress on the lab(s) and homework.
    \item Monday\&Wednesday: Class session with a conceptual review of the chapter material.
    \item Tuesday: Lab activities designed to help gain familiarity with the technical aspects of using R, and RStudio.
    \item Monday\&Wednesday: Office hours.
    \item Friday: Complete DataCamp lab before midnight on Friday.
    \item Sunday: Complete the homework that is due before midnight on Sunday.
\end{itemize}





\vspace{0.5cm}
\large \textbf{How to Prepare:}\\[0.5em]
We will be using three different online tools for this class:
\begin{itemize}
    \item Class Canvas Websites: \href{https://canvas.duke.edu/courses/64773}{https://canvas.duke.edu/courses/64773}. I will use Canvas for posting assignments, class announcements, grades, and other materials, so please make sure to read these messages carefully and check Canvas regularly.
    \item Programming Tools: We will use R statistical programming language for conducting statistical investigations. 
    \begin{itemize}
        \item Language: R. Please download R from \href{https://cran.r-project.org/}{here}.
        \item Environment: RStudio. After installing R, please download RStudio from \href{https://posit.co/download/rstudio-desktop/}{here}.
        \item Publishing System: Quarto. Fter installing RStudio, you can install Quarto by typing "quarto install tinytex" in the terminal.
    \end{itemize}
    \item Online Learning Platform: DataCamp. DataCamp has a lot of very useful tutorials that will help you learn how to code in R. **Note: you must sign up with your @dukekunshan.edu.cn email otherwise you will not receive the free access provided to us. You can access DataCamp from \href{https://www.datacamp.com/}{here}. 
\end{itemize}




\vspace{0.5cm}
\large \textbf{Required Textbook:}\\[0.5em]
\href{https://www.pearson.com/en-us/subject-catalog/p/intro-stats/P200000006161/9780137374922?srsltid=AfmBOorburBAh8Z5hPZXmajuzatW53zDClHAQe8Ks9Fk1pQrkGpSkBuC}{Intro Stats}, 6th Edition by Richard D. De Veaux, Paul F. Velleman, and David E. Bock.





\vspace{0.5cm}
\large \textbf{Assignments:}
\begin{itemize}
    \item Pop Quizzes (10\%): There will be 10 equally weighted unannounced in-class quizzes throughout the semester. These quizzes are designed to encourage participation in the class and help you stay on top of the material. You get full points if $\geq$60\% of questions are answered correctly. I will drop your two lowest quiz grades.
    \item Unit 1\&2 homework (35\%): At the end of the first two groups of content, a homework will be assigned that
    will ask you to analyze a dataset and answer questions related to that concept group. These assignments are always due the Sunday at 11:59 pm China time. 
    \begin{itemize}
        \item Unit 1 homework: 15\%.
        \item Unit 2 homework: 20\%.
    \end{itemize}
    \item Unit 3 exam (20\%): Unit 3 comprehension is better checked through an in-class exam. The exam will be open notes but closed book and will take place during the normal lab session that week.
    \item DataCamp Labs (10\%): There will be 5 equally weighted online DataCamp labs introducing you to key components and packages in R. These assignments are always due Friday at 11:59 pm China time. 
    \item Individual Final Project (25\%): The final project is a chance to show that you can apply the statistical skills we have learned to real-world questions. You will be given a dataset and submit a research report, which has 8-12 pages, double-spaced, including tables and figures. It is due on December 9th (Tuesday) at 11:59 pm China time. 
    \item Extra credit maximum 5\%: Extra credit opportunities may be offered occasionally. These are optional and designed to reward participation, but they will not significantly affect your final grade.
\end{itemize}


\vspace{0.5cm}
\large \textbf{Late Submissions:}
\begin{itemize}
    \item Pop Quizzes \& Individual Final Project: No late submission is allowed.
    \item Other Assignments: Late submission is accepted within 48 hours of the deadline with a penalty. For each 24-hour period after the deadline, a 10\% flat deduction will be applied to the total possible points for that assignment, regardless of your earned score. After 48 hours, the submission will not be accepted. \footnote{For example, the total possible points for unit 2 homework are 20\%. If you submit it 10 hours after the deadline, you will receive a 20\%*10\%=2\% deduction. If you submit it 28 hours after the deadline, you will receive a 20\%*(2*10\%)=4\% deduction. If you submit it 48 hours after the deadline, your grade is 0.}
\end{itemize}



\vspace{0.5cm}
\large \textbf{Attendance Policy:}\\[0.5em]
I do not take attendance. However, if you are not in class for the pop quiz, there is no opportunity to make it up. I urge you not to abuse the opportunity to miss a class without penalty unless your situation absolutely requires to. Topics in this course are interrelated and build on each other. So, missing one class has a potential to set you back in future classes. I strongly encourage you to review the lecture slides and read the textbook if you missed a class.


\vspace{0.5cm}
\large \textbf{Contact Policy:}\\[0.5em]
Please contact me via email or Canvas message. I generally respond within 48 hours. However, during busy times, especially when deadlines are approaching, I may receive a high volume of messages, so please allow extra time for a reply.

\vspace{0.5cm}
\large \textbf{AI Policy:}\\[0.5em]
You may use AI tools (such as ChatGPT, DeepSeek, or similar) to support your learning in this course. However, all submitted work must be your own original writing and analysis. You may not copy and paste AI-generated content into your final project report, quizzes, or other graded assignments. If you use AI tools to brainstorm or check your work, you are responsible for verifying the accuracy and originality of your final submission.



\vspace{0.5cm}
\large \textbf{Academic Integrity:}\\[0.5em]
In addition to skills and knowledge, DKU aims to teach you appropriate ethical and professional standards. You will find extensive information on academic integrity \href{https://ugstudies.dukekunshan.edu.cn/academic-integrity/}{here}.

In line with these policies, dishonesty of any kind will not be tolerated in this course. Dishon- esty includes, but is not limited to, cheating, plagiarizing, fabricating information, facilitating acts of academic dishonesty by others, having unauthorized possession of examinations, sub- mitting work of another person or work previously used without informing the instructor, or tampering with academic work of other students. Whenever in doubt, ask me about ap- propriateness of your actions.

\vspace{0.5cm}
\large \textbf{Disability Policy:}\\[0.5em]
If you need an accommodation due to a disability, you should not hesitate to request one. The process is that requests should be sent to the Dean of Undergraduate Studies, who will contact me with recommended type of accommodation that is needed. You do not need to disclose your reason for requesting an accommodation with me, and asking through the Dean of Undergraduate Studies helps make things official for both you and me.







\newpage
\Large \begin{center}
\textbf{Tentative Schedule}\footnote{Please notice that this schedule is tentative and subject to changes. All changes will be announced via Canvas.}
\end{center}

\Large \textit{\textbf{Unit 1: Distributions}} \\

\large \textbf{Week 1}
\begin{itemize}
    \item Lecture 1.1: Intro to the course (Monday, Oct 20)
    \begin{itemize}
        \item Reading: Chapter 1, 2.1 and 2.2, and 3
        \item Topics:
        \begin{itemize}
            \item Syllabus
            \item What are data and variables?
            \item How to display quantitative and qualitative variables?
            \item Contingency tables
        \end{itemize}
    \end{itemize}
    \item Lab 1: R and Quarto familiarization (Tuesday, Oct 21)
    \item Lecture 1.2: Characteristics of distributions (Wednesday, Oct 22)
    \begin{itemize}
        \item Reading: Chapter 2.3-5 and 4
        \item Topics:
        \begin{itemize}
            \item How to describe the shape, center, and spread of a distribution?
            \item How to compare distributions?
            \item Dealing with problems (outliers, reexpression)
        \end{itemize}
    \end{itemize}
    \item To-dos: 
    \begin{itemize}
        \item DataCamp lab 1: Introduction to R and Tidyverse (due Friday, Oct 24 at 11:59 pm)
    \end{itemize}
\end{itemize}



\textbf{Week 2}
\begin{itemize}
    \item Lecture 2.1: Normal distribution (Monday, Oct 27)
    \begin{itemize}
        \item Reading: Chapter 5
        \item Topics: 
        \begin{itemize}
            \item Standard deviation and standardizing values
            \item Normal models
            \item Normal percentiles
        \end{itemize}
    \end{itemize}
    \item Lab 2: Advanced Quarto editing (Tuesday, Oct 28)
\end{itemize}




\Large \textit{\textbf{Unit 2: Relationships between variables}} \\

\large \textbf{Week 2}
\begin{itemize}
    \item Lecture 2.2: Association and correlation (Wednesdat, Oct 29)
    \begin{itemize}
        \item Reading: Chapter 6
        \item Topics:
        \begin{itemize}
            \item Scatterplots
            \item Correlations
            \item Does correlation imply causation?
        \end{itemize}
    \end{itemize}
    \item To-dos:
    \begin{itemize}
        \item DataCamp lab 2: Introduction to Data Visualization with ggplot2 (due Friday, Oct 31 at 11:59 pm)
        \item Homework assignment 1 (due Sunday, Nov 2 at 11:59 pm)
    \end{itemize}
\end{itemize}





\textbf{Week 3}
\begin{itemize}
    \item Lecture 3.1: Simple linear regression (Monday, Nov 3)
    \begin{itemize}
        \item Reading: Chapter 7
        \item Topics:
        \begin{itemize}
            \item Line of best fit: least squares
            \item The linear model
            \item What are residuals?
            \item Regression assumptions
        \end{itemize}
    \end{itemize}
    \item Lab 3: Working with regressions using dplyr (Tuesday, Nov 4)
    \item Lecture 3.2: Regression wisdom (Wednesday, Nov 5)
    \begin{itemize}
        \item Reading: Chapter 8
        \item Topics:
        \begin{itemize}
            \item Beware extrapolation
            \item Outliers and leverage
            \item Lurking variables
            \item Straightening scatterplots
        \end{itemize}
    \end{itemize}
    \item To-dos:
    \begin{itemize}
        \item DataCamp lab 3: Intermediate Data Visualization with ggplot2 (due Friday, Nov 7 at 11:59 pm)
    \end{itemize}
\end{itemize}



\textbf{Week 4}
\begin{itemize}
    \item Lecture 4.1: Multiple regression (Monday, Nov 10)
    \begin{itemize}
        \item Reading: Chapter 9
        \item Topics: 
        \begin{itemize}
            \item What is multiple regression?
            \item Interpreting multiple regression coefficients
            \item Partial regression plots
            \item Indicator variables
        \end{itemize}
    \end{itemize}
    \item Lab 4: Interpreting coefficients (Tuesday, Nov 11)
    \item Lecture 4.2: Confidence intervals for proportions (Wednesday, Nov 12)
    \begin{itemize}
        \item Reading: Chapter 13
        \item Topics:
        \begin{itemize}
            \item What is a sampling distribution?
            \item When does the normal model apply?
            \item Constructing a confidence interval
            \item Interpreting a confidence interval
        \end{itemize}
    \end{itemize}
    \item To-dos:
    \begin{itemize}
        \item DataCamp lab 4: Modeling with Data in the Tidyverse (due Friday, Nov 14 at 11:59 pm)
        \item Homework assignment 2 (due Sunday, Nov 16 at 11:59 pm)
    \end{itemize}
\end{itemize}




\Large \textit{\textbf{Unit 3: Measuring uncertainty}} \\

\large \textbf{Week 5}
\begin{itemize}
    \item Lecture 5.1: Confidence interval for means (Monday, Nov 17)
    \begin{itemize}
        \item Reading: Chapter 14
        \item Topics: 
        \begin{itemize}
            \item The Central Limit Theorem
            \item Confidence interval for means
            \item Interpreting a confidence interval
            \item Final thoughts on confidence intervals
        \end{itemize}
    \end{itemize}
    \item Lab 5: Bootstrapping (Tuesday, Nov 18)
    \item Lecture 5.2: Hypothesis testing (Wednesday, Nov 19)
    \begin{itemize}
        \item Reading: Chapter 15
        \item Topics:
        \begin{itemize}
            \item What are hypotheses?
            \item P-values
            \item How to make decisions based on p-values?
        \end{itemize}
    \end{itemize}
    \item To-dos:
    \begin{itemize}
        \item DataCamp lab 5: Your choice of any DataCamp course (due Friday, Nov 21 at 11:59 pm)
    \end{itemize}
\end{itemize}

\textbf{Week 6}
\begin{itemize}
    \item Lecture 6.1: Hypothesis testing wisdom (Monday, Nov 24)
    \begin{itemize}
        \item Reading: Chapter 16
        \item Topics: 
        \begin{itemize}
            \item Interpreting p-values
            \item Alpha and critical values
            \item Practical vs. statistical significance
            \item Type I and II errors
            \item Power of a test
            \item Ethical issues
        \end{itemize}
    \end{itemize}
    \item Lab 6: Unit 3 exam (Tuesday, Nov 25)
\end{itemize}





\Large \textit{\textbf{Unit 4: Statistical inference}} \\

\large \textbf{Week 6}

\begin{itemize}
    \item Lecture 6.2: Comparing groups (Wednesday, Nov 26)
    \begin{itemize}
        \item Reading: Chapter 17
        \item Topics:
        \begin{itemize}
            \item Confidence intervals for comparing two samples
            \item Assumptions and conditions for two-sample hypothesis tests
            \item Two-sample z test
            \item Two-sample t test
        \end{itemize}
    \end{itemize}
    \item To-dos:
    \begin{itemize}
        \item Homework assignment 3 (due Sunday, Nov 30 at 11:59 pm)
    \end{itemize}
\end{itemize}

\textbf{Week 7}
\begin{itemize}
    \item Lecture 7.1: Returning to regression (Monday, Dec 1)
    \begin{itemize}
        \item Reading: Chapter 20
        \item Topics:
        \begin{itemize}
            \item Regression inference and intuition
            \item The regression table
            \item Confidence and prediction intervals
        \end{itemize}
    \end{itemize}
    \item Lab 7: Independent learning for individual final project (Tuesday, Dec 2)
    \item Lecture 7.2: Interpretation activity \& model building practice (Wednesday, Dec 3)
    \begin{itemize}
        \item Topics:
        \begin{itemize}
            \item How to read academic statistical results
            \item Locating the model
            \item Interpreting the test
            \item Determining possible weaknesses of the model
        \end{itemize}
    \end{itemize}
    \item To-dos: 
    \begin{itemize}
        \item Individual final project (due Tuesday, Dec 9 at 11:59 pm)
    \end{itemize}
\end{itemize}
\end{document}