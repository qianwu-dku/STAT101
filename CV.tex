\documentclass{resume} % Use the custom resume.cls style

\usepackage[left=0.4 in,top=0.4in,right=0.4 in,bottom=0.4in]{geometry} % Document margins
\name{Qian Wu} % Your name
% You can merge both of these into a single line, if you do not have a website.
\address{\href{mailto:qian.wu2@dukekunshan.edu.cn}{Email: qian.wu2@dukekunshan.edu.cn} \hspace{3mm} \href{https://qian-wu.weebly.com/}{Personal Website: https://qian-wu.weebly.com/}}

\begin{document}

%----------------------------------------------------------------------------------------
%----------------------------------------------------------------------------------------
% EDUCATION SECTION
%----------------------------------------------------------------------------------------

\begin{rSection}{Academic Positions}
{\bf Assistant Professor of Economics}, Duke Kunshan University \hfill {2025/8-}\\
{\bf Assistant Professor of the Practice}, Duke University \hfill {2025/8-}
\end{rSection}


\begin{rSection}{Education}

{\bf PhD in Economics}, Indiana University \hfill {2025/5}\\
{\bf MS in Economics}, Baylor University \hfill {2019/5}\\
{\bf BS in Finance}, Zhengzhou University \hfill {2016/7}
%Minor in Linguistics \smallskip \\
%Member of Eta Kappa Nu \\
%Member of Upsilon Pi Epsilon \\


\end{rSection}



\begin{rSection}{RESEARCH INTERESTS}
Macro-Finance, Macroeconomics
\end{rSection}
 


\begin{rSection}{PROFESSIONAL EXPERIENCE}
Summer Intern at Inter-American Development Bank (IDB)
\hfill{2023/6 - 2023/8}
\iffalse
\begin{itemize}
\addtolength\itemsep{-2mm}
    \item Led a comprehensive review of international literature and policies related to sustainable housing settlements, green housing finance arrangements, and subsidy programs.
    \item Organized and participated in meetings with global housing market stakeholders to enhance academic understanding and foster institutional collaboration.
    \item Authored the paper titled "Innovative Disruption to Address Housing Deficits in Latin American and Caribbean Countries (LAC) Considering International Best Practices".
\end{itemize}

\textbf{Research Assistant} for \textbf{Brittany Almquist Lewis} \hfill{2022/8 - 2023/5}
\begin{itemize}
\addtolength\itemsep{-2mm}
    \item Utilized High Performance Computer (HPC) Linux command line programming to access and clean Corelogic mortgage loan origination and MBS data.
    \item Employed fuzzy match techniques in R to establish a crosswalk between mortgage servicers and financial entities.
    \item Analyzed housing market data in STATA using advanced econometric and statistical models, including correlation analysis, fixed effects, static Difference-in-Difference, and stacked dynamic Difference-in-Difference.
\end{itemize}
\fi
\end{rSection} 


\begin{rSection}{WORKING PAPERS}
    
    \textit{\textbf{A New Keynesian Preferred Habitat Model with Repo} } (Job Market Paper)
    \begin{itemize}
        \item[]\textbf{Abstract:} This paper documents puzzling discrepancies in the Treasury cash and repo markets during the Global Financial Crisis (GFC) and the Covid-19 pandemic. To explain these observations, I develop a New Keynesian Preferred Habitat model with repo featuring market segmentation, financial frictions, and preference shocks. The stochastic discount factor captures both financial and macroeconomic conditions. In this framework, financial market tensions can trigger real recessions, even in the absence of fundamental disruptions. The model illustrates a flight-to-liquidity demand during the GFC, and a flight-from-safety supply during the Covid-19 pandemic. The findings suggest that the effectiveness of monetary policies depends on financial frictions and the relative importance of the cash versus repo borrowing channels. Overall, this paper underscores the strong linkage between financial markets and the real economy.
        \item[]\textbf{Presentations:} IU Macroeconomics Brownbag (x3); Chinese Economists Society (CES) 2024 Annual Conference; Singapore Economic Review Conference (SERC) 2024; Midwest Macroeconomics Meetings (MMM) Fall 2024; 19th Economics Graduate Student Conference; Duke-Kunshan University; American Finance Association (AFA) 2025 Annual Meeting Poster Session; Cornerstone Research; NYU-Shanghai Finance; University of Richmond; Colgate University (cancelled).
    \end{itemize}
   
     

    \item \textit{\textbf{SOFR So Good? New Benchmark Interest Rate and Crowding-Out Effect} } 
    \begin{itemize}
        \item[]\textbf{Abstract:} This paper examines the scarce collateral channel through which government debt may create an additional crowding-out effect on asset prices and macroeconomic variables under the SOFR regime. An increased supply of Treasuries diminishes their scarcity value, leading to higher borrowing costs for Treasury holders in the repo market and an increase in the SOFR. I provide empirical evidence demonstrating that rising government debt correlates with an increase in SOFR.  I then build a stylized model where LIBOR or SOFR can index the business coupon rate. This scarce collateral channel allows public debt to impact the real economy under the SOFR regime without relying on distortionary taxes, though quantitative analysis indicates that the effect in general equilibrium is minimal.
        \item[]\textbf{Presentations:} IU Macroeconomics Brownbag (x2); 2022 Hoosier Economics Conference; 2023 AEA CSWEP Wrokshop.
    \end{itemize}
\end{rSection} 


\newpage
\begin{rSection}{WORK IN PROGRESS}
\textbf{\textit{Price Implications of ESG Mandates Across Market Structures}} (with Christian Heyerdahl-Larsen)
\end{rSection}



\begin{rSection}{HONORS AND AWARDS}
Best Graduate Student Paper  \hfill{Hoosier Economics Conference, 2022} \\
Best Third-Year Paper \hfill{IU Economics Department, 2022} \\
Graduate Student Travel Award \hfill{IU College of Arts and Sciences, 2024} \\
AFA PhD Student Travel Grant \hfill{AFA, 2024} 
\end{rSection} 


\begin{rSection}{PRESENTATIONS}
 (*:scheduled) 

\textbf{2022}: IU Hoosier Economics Conference; IU Macro Brownbag. \\
\textbf{2023}: 2023 American Economic Association (AEA) CSWEP Wrokshop; Inter-American Development Bank (x2); IU Macro Brownbag (x2). \\
\textbf{2024}: IU Macroeconomics Brownbag (x2); Chinese Economists Society (CES) 2024 Annual Conference; Singapore Economic Review Conference (SERC) 2024; Midwest Macroeconomics Meetings (MMM) Fall 2024; 19th Economics Graduate Student Conference; Duke-Kunshan University. \\
\textbf{2025}: American Finance Association (AFA) 2025 Annual Meeting Poster Session; Cornerstone Research; NYU-Shanghai Finance; University of Richmond; Colgate University (cancelled).
\end{rSection} 



\begin{rSection}{TEACHING EXPERIENCE}
Instructor of 
\begin{itemize}
\addtolength\itemsep{-2mm}
    \item[]  Statistical Analysis for Business and Economics \hfill{2021 spring, 2022 spring, 2022 fall, 2023 spring}
    \item[] Money and Banking \hfill{2024 fall}
\end{itemize}  
Teaching Assistant of 
\begin{itemize}
\addtolength\itemsep{-2mm}
    \item[] Intro to Macroeconomics \hfill{2019 fall, 2020 spring}
    \item[] Foundation of Economics for Business I \hfill{2020 fall, 2021 summer}
    \item[] Foundation of Economics for Business II \hfill{2024 spring}
    \item[] Statistical Analysis for Business and Economics \hfill{2021 fall, 2023 fall} 
\end{itemize}
\end{rSection} 

\begin{rSection}{SKILLS}
\textbf{Programming} MATLAB, Dynare, R, STATA, Python
\end{rSection}

\iffalse
\begin{rSection}{MISCELLANEOUS}
    \begin{itemize}
        \item[]Date of Birth: 1994/10/17 
        \item[]Chinese citizen, F1 Visa holder
    \end{itemize}
\end{rSection}
\fi



\begin{rSection}{REFERENCES}
    Bulent Guler  \hspace{7.5cm} Christian Heyerdahl-Larsen\\
    Department of Economics, Indiana University \hspace{2cm} Department of Finance, BI Norwegian Business School\\
    bguler@iu.edu \hspace{7.35cm} christian.heyerdahl-larsen@bi.no\\
\\
Rupal Kamdar \hspace{7.2cm} Nastassia Krukava (teaching)\\
Department of Economics, Indiana University \hspace{2cm} Department of Economics, Indiana University\\
rkamdar@iu.edu \hspace{7cm} nkrukava@iu.edu 
\end{rSection}





\hfill Last updated: May 2025

\end{document}
